\documentclass[12pt, a4paper]{scrartcl}
\usepackage[utf8]{inputenc}

\usepackage{graphicx}
\usepackage{fancyhdr}
\usepackage{url}
\usepackage{listings}
\usepackage{booktabs}
\usepackage{palatino,avant}
\usepackage{amsmath,amssymb,amsthm,mathabx,mathrsfs,stmaryrd}
%\usepackage{txfonts}
\usepackage{pb-diagram}
\usepackage[pdftex,bookmarks=true,bookmarksnumbered=true,bookmarksopen=true,colorlinks=true,filecolor=black,
                linkcolor=red,urlcolor=blue,plainpages=false,pdfpagelabels,citecolor=black,
                pdftitle={VSY-Dokumentation},pdfauthor={Michael Stapelberg}]{hyperref}
\begin{document}
\pagestyle{fancy}
\chead{VSY}
\lhead{}
% New paragraph
\newcommand{\np}{\bigskip\noindent}

% No indention at new paragraphs
\setlength{\parindent}{0pt}

\section{Architektur}

Als Architektur haben wir uns für ein Client/Server-Modell entschieden, da es
deutlich einfacher zu implementieren ist als ein Peer-to-peer-Netz. Bei
letzterem gibt es das Problem des Nachrichtentransports: Entweder müssen alle
Peers sich mit allen anderen verbinden um in der Lage zu sein, Nachrichten an
alle zu senden (Full-Mesh) oder man braucht eine Form der
Nachrichtenweitergabe. Bei einem Client/Server-Modell entfällt diese
Problematik. Weiterhin ist bei dieser Architektur der Bezug zur tatsächlichen
Praxis höher, bei der Peer-to-Peer-Netze selten eingesetzt werden\footnote{Eine
bekannte Ausnahme stellt World of Warcraft mit dem BitTorrent-gestützten Update
Downloader dar}.

\subsection{Basis-Protokoll: HTTP}

Das grundlegende Protokoll, welches wir verwenden, ist HTTP. In der Praxis
verwendet man aus mehreren Gründen gerne HTTP:
\begin{itemize}
	\item Das Protokoll verwendet ASCII (statt Binärdaten) in einem
	einfachen, für Menschen lesbaren, Format.

	\item Implementationen sind in jeder ernstzunehmenden
	Programmiersprache ver\-füg\-bar und in der Regel weitreichend getestet.

	\item HTTP ist in den meisten Firewalls freigeschaltet, weil das World
	Wide Web eine so weit verbreitet Technologie des Internets darstellt.

	\item Zusätzliche Software wie Load-Balancer, Proxies, etc. sind für
	HTTP bereits verfügbar (auch als direkt in Hardware implementierte und
	somit sehr performante Varianten).
\end{itemize}

\subsection{Serialisierung: JSON}

Zum Serialisieren von Daten verwenden wir für beide Seiten (also für Requests
vom Client an den Server und für Antworten des Servers zum Client) die
JavaScript Object Notation (JSON).
\np

Der geringe Umfang von JSON (Arrays, Maps/Hashes, Strings, Numbers) macht es
einfach, entsprechende Bibliotheken in allen ernstzunehmenden
Programmiersprachen zu implementieren. Außerdem ist JSON (bei geeigneter
Formatierung) leicht für Menschen lesbar. Im Gegensatz zu XML ist der Overhead
sehr gering.

\subsection{Protokoll}

Mithilfe von HTTP und JSON haben wir eine Protokolldefinition erstellt (siehe
Datei \texttt{PROTOCOL}), welche festhält, welche URLs mit welchen Parametern
aufgerufen werden, was dieser Aufruf bewirkt und mit welcher Antwort man zu
rechnen hat.

\clearpage

\section{Server}

Wir haben den Server in Perl implementiert, da Perl eine moderne Sprache ist,
die es einem ermöglicht, in kurzer Zeit sauber strukturierte und performante
Programme zu schreiben.
\np

Auf folgende Module haben wir aufgebaut:
\begin{description}
	\item[Moose] Ein modernes Objektsystem für Perl. Vereinfacht die
	Implementation von objekt-orientierten Modulen

	\item[Tatsumaki] Ein asynchrones und hoch-performantes Web-Framework
	(auf Basis des Web-Stacks Plack und dem Event-basierten AnyEvent)

	\item[JSON::XS] Ein Modul zum Enkodieren/Dekodieren von JSON,
	implementiert in C für hohe Geschwindigkeit.

	\item[Test::More] Ein Modul zum Implementieren von Test-Cases.
\end{description}

Wie in der Perl-Welt üblich sind wir nach dem Modell des \textsc{Test Driven
Development} vorgegangen. Das heißt, wir haben zunächst die Testcases
implementiert (siehe Verzeichnis \texttt{t/}) und anschließend die
Implementation so durchgeführt, dass alle Tests erfolgreich ausgeführt werden.
\np



\end{document}
